
\documentclass[preprint,12pt]{elsarticle}
\usepackage{etex}


\makeatletter
\providecommand{\doi}[1]{%
	\begingroup
	\let\bibinfo\@secondoftwo
	\urlstyle{rm}%
	\href{http://dx.doi.org/#1}{%
		doi:\discretionary{}{}{}%
		\nolinkurl{#1}%
	}%
	\endgroup
}
\makeatother
%\usepackage{natbib}
% encoding and languages
\usepackage{hyperref}
\usepackage[utf8]{inputenc}
\usepackage[T1]{fontenc}
\usepackage[ngerman,english]{babel}
\usepackage{wrapfig}
\usepackage{times}
\usepackage{paralist}
\usepackage{graphicx, color}
%\usepackage{subfigure}
\usepackage{subfig}
%\usepackage{subfloat}
\usepackage{booktabs}
\usepackage{caption}
%\usepackage{subcaption}
\usepackage{paralist}
\usepackage{amssymb, amsfonts}
\usepackage{amsmath}
%\usepackage{amsthm}
\usepackage{amsopn}
\newtheorem{myProb}{Problem}
\newtheorem{mythm}{Theorem}

\newtheorem{myproof}{Proof}
\newtheorem{mysketch}{Proof Sketch}
%\usepackage[ruled]{algorithm2e}
%\usepackage[colorinlistoftodos]{todonotes}
%\usepackage{qtree}
\usepackage{tikz, tikz-qtree}
\usepackage[ruled]{algorithm2e}
%\usepackage{Algorithmic}
\usepackage{times}
\usepackage{microtype}
\usepackage{url}
\usepackage{balance}
\let\proof\relax 
\let\endproof\relax
\usepackage{multirow}
\usepackage{relsize}
\usepackage{caption}
\usepackage{color, colortbl}
\usepackage{array, booktabs, ragged2e}
\usepackage{makecell}
\usepackage{tikz}
\usepackage{textpos}

\usepackage{pdflscape}
\newcommand{\RNum}[1]{\uppercase\expandafter{\romannumeral #1\relax}}

\graphicspath{{./Figures/}}

\newcommand{\mypar}[1]{\smallskip\noindent\textbf{#1.}}

\newtheorem{prob}{Problem}


\newtheorem{mydef}{Definition}
%\newtheorem{remark}{Remark}

\newcommand{\todo}[1]{{\textcolor{red}{\bf {#1}}}}

%\newcommand{\mytitle}{Exploiting Software and Hardware Advances  in Optimization Methods for Improved Wind Farm Design}
\newcommand{\mytitle}{Exploiting Hardware and Software Advances 
 for Quadratic Models of Wind Farm Layout Optimization}
%\newcommand{\mytitlerunning}{\mytitle}
%\newcommand{\myauthor}{Arik Senderovich, Eldan Cohen, and J. Christopher Beck}
%\newcommand{\myauthorhyperref}{Arik Senderovich, Eldan Cohen, and J. Christopher Beck}


%\hypersetup{
%    bookmarks=true,        % show bookmarks bar?
%    unicode=true,          % non-latin characters in bookmarks?
%    pdffitwindow=false,    % window fit to page when opened
%    pdfstartview={FitH},   % fits width of the page to the window
%    pdftoolbar=true,       % show acrobat toolbar?
%    pdfmenubar=true,       % show acrobat menu?
%    colorlinks=true,      % false: boxed links; true: color links
%    linkcolor={blue!80!black},       % color of internal links
%    urlcolor={blue!30!black},        % color of external links
%    citecolor={green!60!black},       % color of links to bibliography
%    filecolor=black,       % color of file links
%    pdftitle={\mytitle},           % title
%    pdfauthor={\myauthorhyperref}, % author
%    pdfcreator={\myauthorhyperref} % creator of pdf
%}

\begin{document}
	

\title{\mytitle}

%\subtitle{Extended Abstract}
%\author{}
%\institute{}
%\author{\myauthor}
%\institute{Department of Mechanical and Industrial Engineering, University of Toronto\\
%\email{sariks@mie.utoronto.ca, ecohen@mie.utoronto.ca, jcb@mie.utoronto.ca}}


\author[add1]{Arik Senderovich}

%\ead[url]{https://ischool.utoronto.ca/profile/arik-senderovich/}
\ead{arik.senderovich@utoronto.ca}

\author[add2]{Eldan Cohen}
\ead{ecohen@mie.utoronto.ca}
%\ead[url]{homepageldan}




%\address[1]{}



%\address[2]{40 St. George St., Toronto, Canada}
\author[add2]{Jiachen Zhang}
\ead{jasonzjc@mie.utoronto.ca}

\author[add2]{J. Christopher Beck\corref{cor1}}
\ead{jcb@mie.utoronto.ca}


%\ead[url]{homepageldan}
%\address[3]{}
\address[add1]{Faculty of Information, University of Toronto, 140 St. George St., Toronto, Canada}
\cortext[cor1]{Corresponding author}
\address[add2]{Department of Mechanical and Industrial Engineering, University of Toronto, 40 St. George St., Toronto, Canada}

\begin{abstract} 
%Wind farms play an increasingly important
%role as a source of renewable energy and 
  A key aspect of the design of a wind farm is the wind farm layout optimization (WFLO) problem: given a wind farm site and a model of the wind patterns, decide the location of individual wind turbines to maximize energy production subject to proximity restrictions and wake-based interference among turbines. Given the pairwise wake interactions, it is natural
%While a number of heuristic and exact techniques have been proposed to solve WFLO, it is common
to model the energy objective as a quadratic function as, indeed, has been done in some existing optimization models. Recent advances in both optimization hardware and software have targeted quadratic constraints: commercial mixed integer linear solvers have been extended to address some quadratic problems and nascent specialized hardware, including quantum computing systems, have focused on solving quadratic unconstrained binary optimization (QUBO) problems. 
%It is a key problem 
%when designing a new wind farm, as it has a direct influence
%on the amount of the total produced energy. 
%In recent years there have been advances in both software technologies (e.g., the extension of commercial mixed integer linear programming solvers to handle quadratic problems) and nascent hardware development (e.g., quantum computing chips and specialized classical chips) that have relevance for the WFLO problem. In both cases, the advances have 
%Recent years saw 
%advances in software and hardware optimization technologies, such as Gurobi's quadratic solver 
%and Fujitsu's digital annealer (DA), respectively. 
%These technologies create an opportunity to improve 
%over current wind farm optimization solutions
%both in terms of solution quality and solve time. 
%In order to exploit these technologies, 
%we must represent the WFLO problem as 
%a quadratic optimization model. Specifically, in order 
%to apply a software-based optimization
%solution (Gurobi), 
In this paper, we 
introduce two novel quadratic programming models for WFLO: a quadratic constrained optimization problem (QCOP) with binary decision variables and a QUBO.
%Subsequently, we provide the unconstrained counterpart of the QCOP, 
%namely the quadratic unconstrained binary optimization (QUBO), which  
%is the underlying representation for quantum-inspired 
%optimization hardware such as the DA. 
A thorough empirical 
evaluation using a commercial solver and specialized QUBO hardware
%demonstrates the strengths of the two models. 
%Our results
show that both approaches yield fast, high-quality solutions 
that
improve the state of the art. In particular, the QUBO model delivers high quality solutions very quickly while the QCOP model can be used to find better solutions and provide quality guarantees over a longer run-time.
\end{abstract}
\begin{keyword}
	Wind Farm Layout Optimization (WFLO) \sep Quadratic Programming (QP) \sep Quadratic Unconstrained Binary Optimization (QUBO) \sep Digital Annealer
\end{keyword}

\maketitle 

\section{Introduction}
% Todo
The goal in the wind farm layout optimization (WFLO)
problem~\cite{MOSETTI1994105} is to locate wind turbines within a
predefined area to maximize the total power harvested from the wind
stream.  While physical considerations based on the diameter of the
blades prevent turbines being placed too close to each other, turbine
location decisions influence the total power production due to
interference effects (i.e., wakes) generated by upstream turbines.
These inter-turbine effects are best captured by Jensen's wake model~\cite{jensen1983note} when superimposed
using a sum-of-squares expression~\cite{Zhang2014}. However,
this wake modeling yields intractable optimization problems
that must be approximated using either quadratic or linear functions~\cite{turner2014new}.
An alternative formulation of WFLO relies on a less accurate wake model, which uses linear superpositioning
to capture multiple wakes~\cite{donovan2005wind}.
The latter can be captured using quadratic
constraints and objective functions. However, exact optimization
approaches that have been proposed for the alternative WFLO formulation~\cite{Zhang2014,kuo2016wind,donovan2005wind,fagerfjall2010optimizing,archer2011wind,sorkhabi2018constrained}
represent the quadratic relations using approximate linear models,
%that approximate wake dependencies,
as solving quadratic programming problems
is computationally challenging compared to linear models.

In this work, we introduce two novel quadratic WFLO models. The first
model represents WFLO as a quadratic constrained optimization
problem (QCOP).  The QCOP can be solved using state-of-the-art mathematical solvers
that have been recently extended for such problems, e.g., Gurobi~\cite{gurobi}.  Our
second model is a quadratic unconstrained binary optimization
(QUBO) representation of the WFLO problem.  Such a formulation enables the use of
nascent specialized optimization hardware tailored to solve QUBOs: we use Fujitsu's Digital Annealer (DA) \cite{aramon2019physics} here.
%, a hardware-based
%QUBO solver, to demonstrate the strength of our approach.

The main
contributions of this paper are:
\begin{enumerate} 
\item We propose a novel quadratic modeling framework for WFLO that encompasses both constrained and unconstrained quadratic optimization models.
 %that captures wake 
%effects directly through their linear superposition. 
%without the need for approximations and ad-hoc solutions,
\item We prove that the QCOP variation of WFLO is $\mathcal{NP}$-hard.
\item Through numerical experiments on two instantiations of our quadratic framework using Fujitsu's Digital Annealer and a commercial software-based quadratic solver (Gurobi), we demonstrate state-of-the-art performance in, respectively, quickly finding high-quality solutions and finding and proving optimal solutions over a longer run-time.
\end{enumerate}

Our experiments are focused on the performance of the two approaches
compared to 
existing optimization methods. To this end,
we experiment with 12 standard WFLO benchmark problems (c.f.~\cite{turner2014new})
and show that solving quadratic WFLO models
using Gurobi and the DA
achieves state-of-the-art performance.
%, often outperforming existing solutions when provided sufficient run-time.
Furthermore, we observe that the DA often provides solutions of
similar quality to software-based approaches but with two orders of
magnitude lower run-time.
%The ability to quickly obtain competitive
%solutions on large WFLO instances can be employed to solve multiple
%wind farm design problems.

The remainder of this paper is structured as follows. Section~\ref{sec:related} presents our wind farm model and a literature review of exact and approximate
optimization solutions to the layout problem. The main contribution, namely the two quadratic models, is introduced in Section~\ref{sec:QUBO4WFLO} and Section~\ref{sec:eval} demonstrates the value of our approach via a thorough empirical evaluation.
%of the proposed methods. 
Lastly, Section~\ref{sec:conclusion} provides concluding remarks and directions for future work.   


\section{Background}
\label{sec:related}

We start this section by presenting the physical model of a wind farm
%used throughout the paper.
%Our wind farm model is
based on the common
notation and description presented in Zhang et al.\ \cite{Zhang2014}.  Subsequently,
we provide a literature review of existing mathematical optimization
methods for solving WFLO, focusing on their similarities and
differences with our approach.
 
\subsection{The Wind Farm Model}

%\paragraph{Wind Farm as a Grid}
We consider a square wind farm that we model as an $n\times n$
grid with $n$ being the number of cells on each axis.  The cells have
an area of $c^2$ (in square meters), i.e., if $c=100$ meters and
$n=20$, the farm area will be $2km\times 2km$.  We
assume that the terrain is flat and that we have no constraints
related to the terrain.  Furthermore, we do not account for the
influence of noise on the wind farm's surroundings. The wind farm
literature treats the more complex farm models that capture non-flat
terrain~\cite{song2015lazy,kuo2016wind}, multi-typed
turbines~\cite{feng2017design}, noise
effects~\cite{Zhang2014,sorkhabi2016impact,yin2014multi}, and complex
objectives (including turbine installation and maintenance
costs)~\cite{lackner2007analytical,sorkhabi2018constrained}.  In this work, we focus on the
novel quadratic approach for WFLO, and hence consider the most
abstract wind farm model. Our model can be extended, in the
future, to include additional problem characteristics.
 
We assume that the number of identical turbines to be placed is known 
in advance and denoted by $m$. To place these $m$ turbines, 
we must consider two types of interference effects, namely \emph{proximity constraints} and \emph{wakes}.
In what follows, we explain the two effects using a toy example of placing a single turbine on an 
$8\times8$ grid (see Figure~\ref{fig:field_model}).

\begin{figure}[t]
	\centering
	\includegraphics[scale = 0.9]{field_model.pdf}

	\caption{Wind farm model: proximity constraints, wakes, wind direction.}\label{fig:field_model}
\end{figure}



\paragraph{Proximity Constraints}
Turbines must be placed at least \emph{five rotor diameters}
apart. Depending on the cell size, we shall formulate the
corresponding constraint as part of the optimization model. For
example, if a turbine is placed in the center of a
cell, a cell size of $c = 100$ meters and a rotor diameter
of $40$ meters, requires that two turbines are not placed in adjacent cells. In Figure~\ref{fig:field_model}, the red area
around the turbine represents the forbidden cells.
	 
\paragraph{Wakes}
Turbines influence each other through \emph{wake} interference effects~\cite{jensen1983note,shakoor2016wake}.
%that are
%referred to as \emph{wakes}~\cite{jensen1983note,shakoor2016wake}.
Wakes cause reduction of the effective power produced by a turbine due
to upstream turbines that change the airflow dynamics. In
Figure~\ref{fig:field_model}, the wind is assumed to be
blowing from left to right. The wake effect due to placing the turbine
expands as shown in the diagram.
%: it will influence only turbines that
%are placed in the orange cells.  In other words,
If a turbine is placed in the orange area, its effective power is
reduced due to the upstream turbine.
% that we placed in the $8\times8$ farm.
Moreover, a given turbine may be placed downstream from multiple turbines and hence
multiple wakes must be taken into account for each location.  The
reduction of power due to the wake effect is determined by three
parameters: (1) the distance between the upstream turbine and the cell
for which we want to compute the effect (denoted by $x$ in
Figure~\ref{fig:field_model}), (2) the radius of the cone's opening at distance $x$ 
(denoted by $r$ in Figure~\ref{fig:field_model}), and (3) the
combination of several wind regimes where a wind regime consists of 
the wind's free speed and direction.  The wake effect can be pre-computed
for each pair of cells since we know: the distance ($x$), the turbine
specifications and terrain conditions (that jointly dictate $r$), and
probabilistic behavior of wind regimes~\cite{Zhang2014}.
	 
\paragraph{Effective Power Calculations in the Presence of Wakes} 
Let $\mathcal{D}$ be the set of wind regimes.
%where every regime, $d\in \mathcal{D}$, is a combination of wind direction and free speed. 
The probability of wind regime 
$d \in \mathcal{D}$ is given by $p_d$ with $\sum_{d \in \mathcal{D}}^{} p_d = 1$. In the experimental section (Section~\ref{sec:eval}),
we provide one example of a probability distribution 
function over $36$ wind regimes (Figure~\ref{fig:prob_wind}). In that example, 
a regime $d=(10^\circ, 12 \mbox{ km/h})$ corresponds to a north-easterly wind blowing at a free velocity of $12\mbox{ km/h}$. The
probability of this regime is $0.008$.

Let $u_{id}$ be the wind velocity at turbine $i = 1,\ldots, m$ for wind regime $d\in\mathcal{D}$.\footnote{Turbines can adjust their orientation according to the current wind 
regime.}  Given the interference of the wakes, the velocity of turbine $i$ is not equivalent to the free wind speed of $d$.
Rather, $u_{id}$, can be computed using the following equation~\cite{Zhang2014}:
\begin{equation}
u_{id} = u_{id\infty} \Bigg[1 - \sqrt{\sum_{j\in\mathcal{U}_{id}} \bigg( 1-\frac{u_{ijd}}{u_{id\infty}} \bigg)^2}  \Bigg], \label{eq:ss}
\end{equation} with $u_{id\infty}$ being the free wind speed (the second component of regime $d$), $\mathcal{U}_{id}$ corresponding to the set of turbines  that are upstream to $i$ 
under wind regime $d$, and $u_{ijd}$ being the wake-reduced speed at $i$ due to upstream turbine $j$. 
The reduced speed, $u_{ijd}$, is computed using the distance $x$ between the two turbines ($i,j$)  and turbine and terrain specifications that yield the cone opening radius, $r$. The set of upstream turbines, $\mathcal{U}_{id}$ is computed  
by rotating the grid wrt to the current wind regime $d \in \mathcal{D}$ and identifying the turbines, $j$, located upstream of turbine $i$ given $d$. The expression in Eq. (\ref{eq:ss}) is referred to as the \emph{sum-of-squares} (SS) model for 
the total expected power in presence of wakes~\cite{Zhang2014}.  We can now write 
the sum-of-squares (SS) expected power of a wind farm with $m$ turbines:
%~(\cite{Zhang2014}):
\begin{equation}
  E_{SS} = \sum_{i=1}^m \sum_{d\in\mathcal{D}} \frac{1}{3} u_{id}^3p_d.\label{eq:ess}
\end{equation}

  Eq. (\ref{eq:ess}) is considered the most accurate analytical total power expression that accounts for multiple wakes~\cite{jensen1983note}. 
  However, the expression is challenging for exact mathematical optimization techniques due to its complexity (i.e., the cube of a square root).
  %since it is a squared root function of a cubic expression. 
  %Hence, a simplified representation for the total expected energy that considers wakes is required. 

An alternative 
form of the total expected power in the presence of multiple wakes is the linear superposition (LS) expression:

\begin{equation} \label{eq:ls}
E_{LS} = \sum_{i=1}^m \sum_{d\in\mathcal{D}} \Bigg(\frac{1}{3}u_{id\infty}^3 -\sum_{j\in\mathcal{U}_{id}} \frac{1}{3}(u_{id\infty}^3 - u_{ijd}^3)\Bigg).
\end{equation}

Note that $E_{LS}$ has the same parameters as $E_{SS}$ but differs in their relationship.
The LS model is less accurate compared to SS~\cite{Zhang2014}, yet as we will show in Section~\ref{sec:QUBO4WFLO}, it 
 results in a quadratic objective function in our WFLO formulations. Such functions have been the subject of recent research advances~\cite{ku2017hybrid,bian2010ising}, motivating the contributions here.
%, which is a desired property from an optimization standpoint. \todo{[JCB: A bit strange to claim that a quadratic model is ``natural'' in the abstract but then argue that the model we adopt is an approximation but it is good because it is quadratic. Sounds like it is not natural and we choose it due to its optimization properties. Arik: I changed the arguments - please see the abstract and the beginning of the introduction.]}
 

\subsection{Optimization Methods for WFLO}

The allocation of turbines in a grid-like wind farm was first
considered as an optimization problem by~\citet{MOSETTI1994105}. Their
method employs a genetic
algorithm~\cite{davis1991handbook}, an approximate technique
with no guarantees of solution quality but with the goal of quickly finding a high-quality turbine placement.


In subsequent work~\cite{turner2014new,Zhang2014}, 
the WFLO problem was solved using \emph{exact}  
optimization methodologies that guarantee that the returned solution is optimal with respect
to the specified objective or is provably within a given bound of optimal.
%(i.e., maximize total expected power). 
The exact methods were shown to yield higher energy values, yet they often suffered from 
high computational cost and long run-times. In practice, it is often desirable to strike a balance between optimality and computational requirements: fast sub-optimal solutions are useful when solving the wind farm design problem numerous times (e.g., when considering different numbers of turbines and various allowed locations), while optimal solutions yield the best possible placements (admitted by the mathematical model) when longer run-times are available.

In what follows, we provide a literature review that covers both exact and approximate methods for WFLO. We subsequently relate our 
work to both approaches via our proposed quadratic optimization framework for WFLO.\footnote{We review 
only literature that considers the same wind farm model as the one presented at the beginning of this section. For a broader literature review see Zhang et al.\ \cite{Zhang2014}.} 
 
\paragraph{Exact Optimization Methods} 
The WFLO problem can be solved  
to optimality, using
techniques from operations research (e.g., mixed-integer linear programming (MILP), quadratic programming (QP), and constraint programming (CP))~\cite{Zhang2014,turner2014new,donovan2005wind}.
These methods are \emph{exact}, in the sense that they guarantee that the returned solution is indeed
the best solution one can obtain for the given problem. Donovan was the first to solve WFLO using a
MILP approach based on the LS energy expression~\cite{donovan2005wind}. Turner et al.\ \cite{turner2014new} approximated the SS energy expression using a quadratic and a linear approximation. The two resulting methods performed well when
compared to existing approximate and exact techniques.
Another non-linear optimization approach \cite{ulku2019new}, relaxed the binary placement variables to continuous values in $[0,1]$, 
and approximated the constraints such that the resulting variables yield an integer placement.
Their approach provides another approximation to the WFLO problem considered by Turner et al.\ \cite{turner2014new}.

Zhang et al.\ \cite{Zhang2014} developed a constraint programming (CP) approach to the problem variation presented above. CP is an optimization technique that has grown from the AI literature that does not make the assumptions on the functional form of the constraints or objective \cite{rossi2006handbook}. The authors compared existing MILP and QP models to their approach for directly optimizing the SS objective and showed that, while promising for smaller problems, applying CP turned out to be computationally intractable for the larger standard WFLO benchmarks. 

%% The most recent attempt to solve the wind farm variation that we presented 
%% in this section is given in Zhang et al.\ \cite{Zhang2014}. The authors compared
%% previously existing MILP and QP models to their approach for directly optimizing the SS objective. To this end, they used constraint programming (CP), which 
%% is an optimization technique that does not make assumptions on the functional form of the objective.
%% However, applying CP turned out to be computationally intractable for the larger standard WFLO benchmarks. 

In our work, we build upon the approach in Donovan~\cite{donovan2005wind}, yet instead of using a linear version of the LS objective, we 
model WFLO using a quadratic representation. %The difference between our approach and the quadratic model
%proposed in
Unlike Turner et al.\ \cite{turner2014new} and Ulku and Alabas-Uslu\ \cite{ulku2019new}, our model is not an approximation of the SS objective, 
but rather an exact representation of the LS objective. That is, while the previous work has approximated a more accurate physical model, we are exactly representing the less accurate physical model: the approximation arises at a different stage of the modeling pipeline.

While the main advantage of exact techniques
is that they provide guarantees on solution quality, they may take a very long time to  find an optimal solution. 
In fact, Section~\ref{sec:QUBO4WFLO} shows formally that WFLO is computationally hard, a result
that we have not yet encountered in the WFLO literature. Therefore,
when one considers solving multiple wind farm design problems, if a fast solution is of essence,
one may turn to approximate optimization techniques.

\paragraph{Approximate Methods} In contrast to exact methods, approximation algorithms often provide   
quick and, hopefully, sufficiently accurate solutions. Several works have solved 
WFLO using evolutionary (genetic) approaches~\cite{MOSETTI1994105,gonzalez2010optimization,grady2005placement}. 
%These methods use two principles (cross-over and mutation)
%to change existing solutions and improve them over time.
These methods do not guarantee termination 
(i.e., they may run forever without achieving a quality criteria), and even when they do terminate (in practical cases) they do
not guarantee optimal solutions. Hence, one must often
introduce non-quality related stopping criteria, 
thus compromising on the quality of the obtained solution~\cite{davis1991handbook}.
Furthermore, there is no natural way of introducing constraints into genetic methods (see e.g., \cite{sorkhabi2018constrained}), which
leads to a potentially costly feasibility check for every solution found. 

The first attempt to address these limitations 
was made using a local search approach~\cite{ozturk2004heuristic}. 
The solution does not guarantee optimality, 
yet it circumvents the termination and feasibility checking issues of the evolutionary
methods. The local search approach attempts to apply either move, remove or add actions given an incumbent 
feasible solution. When a turbine is moved, removed or added, the new solution is evaluated. The procedure terminates 
when a stopping criteria is reached.
The main limitation of this approach 
%~\cite{ozturk2004heuristic}
is that it cannot perform an action 
that leads to an infeasible state and, therefore, often quickly converges to a sub-optimal solution (local optimum) 
that can potentially be much lower quality than the globally optimal solution~\cite{rivas2009solving}. 

To overcome this limitation, Rivas et al.\ \cite{rivas2009solving} used simulated annealing (SA), 
a neighborhood search method that allows visits to infeasible solutions. 
This creates a well-connected solution space that leads the algorithm to escape 
local optima. The SA method was shown to be superior to the approach of Ozturk \& Norman~\cite{ozturk2004heuristic} 
and to the genetic method proposed in Grady et al.\ \cite{grady2005placement}. A drawback of the SA approach 
is its ad-hoc nature. The various components of the algorithm
 must be tailored to the problem at hand and, as a result, small adjustments of the WFLO setting
 (e.g., adding noise considerations), would require major changes to the SA implementation. 
Additional methods and experimental comparisons between various approximate algorithms are available in Samorani~\cite{samorani2013wind}.

In our work, we also use an annealing based approach (similar to Rivas et al.\ \cite{rivas2009solving}). However,
our methodology is generic since it is based on a declarative problem model and
on the ability of specialized hardware to solve (non-ad-hoc) quadratic problems.

%\paragraph{Quadratic Programming as a Unified Framework for Exact and Approximate WFLO}
%\todo{send a clear message here. Perhaps move to next section}.



%
%
%Specifically, we consider two variations of the quadratic programming paradigm.  
%The first one is referred to as quadratic constrained optimization problems (QCOP). 
%It enables maximizing a quadratic (LS-based) energy expression, while 
%ensuring that WFLO constraints are maintained (turbine proximity and total number of turbines is $m$). 
%The second model is a quadratic unconstrained binary optimization (QUBO), 
%which 
%
%
%On the one hand, exact solvers for quadratic programs (QCOP and QUBO) such as Gurobi~\cite{}, 
%use cutting edge
%operations research methods to enhance performance.
%On the other hand, simulated annealing methods have been recently embedded onto 
%designated processing units. These units use quantum-inspired technology to solve
%complex problems quickly and with minimal degradation in terms of the resulting solution~\cite{}.
%Specifically, in this work, we shall use Fujitsu's digital annealer (DA) to solve WFLO. The DA
%also requires a quadratic representation of the WFLO problem. 
%
%Both of these methods use 
%quadratic optimization models, which we shall present in the next section (Section~\ref{sec:QUBO4WFLO}).
%


\section{Quadratic Models for WFLO}
\label{sec:QUBO4WFLO}

A benchmark study comparing state-of-the-art exact approaches \cite{Zhang2014} to 
approximate methods and found that the two perform in a comparable fashion, without large differences between
the attained energy~\cite{yang2019simulated}. Exact methods tend to perform slightly better
in smaller WFLO benchmarks, while simulated annealing methods dominated for larger standard WFLO benchmarks. 
These results lead to the conclusion that both types of methods (exact and approximate) should be considered. Therefore, we introduce a unified paradigm, namely quadratic programming, that enables
both exact and approximate solutions to WFLO. Further, formulating WFLO using quadratic programming enables us to employ 
recent advanced exact software solutions (e.g., Gurobi optimizer) and approximate hardware developments 
(e.g., Fujitsu's Digital Annealer).
%Thus, 
%our approach `enjoys' the best of both worlds (the exact and the approximate). 

%In this work, we argue that 

%that solves quadratic problems using principles from simulated annealing). 

%% In this section,
%% we formulate the WFLO problem
%% using quadratic programming. Specifically,
We consider two
quadratic optimization problems that represent WFLO: a
quadratic constrained optimization problem (QCOP) and 
a quadratic unconstrained binary optimization (QUBO) problem. 
The former represents WFLO characteristics (proximity and number of turbines)
as hard constraints, while the latter considers these characteristics as soft constraints that are penalized as part of the objective function.

We start by presenting the inputs and decision variables into our optimization approaches.
Subsequently, we introduce the two quadratic formulations of WFLO and 
prove its computational complexity. Lastly, we discuss the details of existing
software and hardware solutions that can be used to solve the two formulations. 

\subsection{WFLO Inputs and Decision Variables}
Given the wind farm problem presented in Section~\ref{sec:related},
let $x_i$ be a binary decision variable that represents whether a 
turbine is positioned at location 
$i \in \mathcal{N}$ with $\mathcal{N}$ being the set 
of $k = n^2$ possible turbine locations (cells in the grid). We 
write $x$ as shorthand for $x = (x_1,\ldots, x_k)$. 
We denote by $\mathcal{E} \subseteq \mathcal{N}\times \mathcal{N}$
the set of location pairs that cannot simultaneously host turbines 
due to proximity constraints. The set $\mathcal{E}$ can be pre-computed 
based on problem specifications.
%For example,
%it is common to assume that two turbines must be placed at least five rotor diameters apart. 

Let $u_{id}$ and $u_{id\infty}$ 
be the wind speeds at turbine location $i \in \mathcal{N}$ for wind regime $d \in \mathcal{D}$
with and without interference from other turbines due to wake effects, respectively. 
Note that here we refer to $i$ as a potential turbine location and not the $i$th turbine as we did in Section~\ref{sec:related}.
Further, we denote by $\mathcal{U}_{id} \subseteq \mathcal{N}$ 
the set of upstream 
turbine locations for wind regime $d$
given that a turbine is placed at cell $i$ (i.e., turbines placed in these locations 
will influence a turbine placed in $i$). Finally,
we denote by $u_{ijd}$ the wind speed at location $j$ due to a single wake from upstream turbine at location $i$ with $j \in \mathcal{U}_{id}$. 
The following function  
represents the total expected energy of a wind farm given placement decisions 
$x$: \begin{equation}
f(x) = \sum_{i \in \mathcal{N}}^{} \sum_{d \in \mathcal{D}}^{} p_d ( \frac{1}{3} \ u_{id, \infty}^3 x_i  - \sum_{j \in \mathcal{U}_{id}}^{} \frac{1}{3}(u_{id, \infty} ^3 - u_{ijd}^3)x_i x_j).   \label{ObjFunc}\\
%&s.t.:& \nonumber\\
%&\mbox{       }& \sum_{i \in \mathcal{N}}^{} x_i = m,\label{Cardinality}\\
%&\mbox{       }& x_i + x_j \leq 1,   \forall (i,j) \in \mathcal{E}, \label{Grid}\\
%&\mbox{       }& x_i \in \{0,1\},     \forall i \in \mathcal{N}.
\end{equation} The objective function corresponds to the linear superposition (LS)
expression (c.f., Eq.~\ref{eq:ls} in Section~\ref{sec:related}). We are now ready to provide the two quadratic programs to represent WFLO.

\subsection{WFLO as a Quadratic Constrained Optimization Problem}

To 
maximize $f(x)$ and satisfy the total number of turbines and proximity 
constraints we write
the following quadratic optimization problem (QCOP):
\begin{eqnarray} \label{QCOP}
&\max_{x_i}^{}& \sum_{i \in \mathcal{N}}^{} \sum_{d \in \mathcal{D}}^{} p_d ( \frac{1}{3} \ u_{id, \infty}^3 x_i  - \sum_{j \in \mathcal{U}_{id}}^{} \frac{1}{3}(u_{id, \infty} ^3 - u_{ijd}^3)x_i x_j)   \\
&s.t.:& \nonumber\\
&\mbox{       }& \sum_{i \in \mathcal{N}}^{} x_i = m,\label{Cardinality}\\
&\mbox{       }& x_i + x_j \leq 1,   \forall (i,j) \in \mathcal{E}, \label{Grid}\\
&\mbox{       }& x_i \in \{0,1\},     \forall i \in \mathcal{N}. \label{Vars}
\end{eqnarray} Constraints~\ref{Cardinality}~and~\ref{Grid} ensure that exactly $m$ turbines are placed on the grid
and enforce proximity constraints between the turbines, respectively. We shall refer to the QCOP
formulation of WFLO as WFL-QCOP. 

WFL-QCOP can be solved using an exact optimization solver, e.g., Gurobi~\cite{gurobi}. 
Similar, linear formulations have been previously proposed and solved using exact solvers~\cite{Zhang2014,donovan2005wind}.
\todo{JCB: We need to formally present the ILP model in the experiment section. We can then discuss the fact that the model is identical - because the linearization gives and exact representation. Should put a forward reference to that model here.}
%\todo{Are these linear formulations ``equivalent'' as you claimed or approximations? If they are mathematically equivalent, why do we think a quadratic model will be any better computationally? Arik: The two formulations are equivalent to WFL-QCOP. However, we compare against the ILP model experimentally and it does worse than WFL-QCOP. Hopefully, Jason's part in Section 3.5.1. based on Wen-Yang's thesis will explain why it does better. Originally, I used the quadratic formulation because it seemed more natural. Semantically, the linear model and the quadratic model are the equivalent imho.}



\subsection{WFLO as a Quadratic Unconstrained Binary Optimization Problem}

Quadratic unconstrained binary optimization problems (QUBOs), as their name implies, do not allow the direct representation of hard constraints. Instead, 
%they only 
%have an objective function that 
%one can to maximize (or minimize) and
problem constraints must be represented as part of the objective function
using penalty terms. 

Let $\lambda = (\lambda_1,\lambda_2)$ 
be a vector of constraint parameters with $\lambda_i >0$. The equivalent QUBO formulation of the QCOP in Eq. (\ref{QCOP}) is given by,
\begin{equation}\max_{x}^{} f(x) - \lambda_1 (\sum_{i \in \mathcal{N}}^{} x_i -m) ^2 - \lambda_2 \sum_{(i,j) \in \mathcal{E}}^{} x_i x_j , \label{QUBO}\end{equation}
with the term $ \lambda_1 (\sum_{i \in \mathcal{N}}^{} x_i -m) ^2$ 
penalizing solutions that violate the exactly $m$ turbines constraint and the term $\lambda_2 \sum_{(i,j) \in \mathcal{E}}^{} x_i x_j$ penalizing
solutions that place turbines too close together. The two penalty terms $\lambda_1$ and $\lambda_2$ 
must be large enough to ensure that the solutions are indeed feasible wrt the problem constraints. We shall refer to the QUBO representation of WFLO as WFL-QUBO.
%These types of constraints
%are referred to as \emph{soft constraints} as they do not guarantee a feasible solution 
%for an arbitrary selection of $\lambda$.   

\subsection{The Computational Complexity of WFL-QCOP}
\label{sec:computational}
%(linear) problems were formulated and solved using  exact solvers.
Despite the existence of mathematical models for WFLO in the literature~\cite{Zhang2014,donovan2005wind} that are expressive enough to represent $\mathcal{NP}$-hard problems, we are unaware of an explicit treatment of the computational complexity of WFLO.
%However, these solvers should only be used if the problem is computationally hard.
%We are unaware of an existing complexity result for the LS variation of WFLO (our QCOP).
We therefore provide a straightforward proof of the computational complexity of WFL-QCOP. 

\begin{mythm}
	The computational complexity of WFL-QCOP defined in \eqref{QCOP}-\eqref{Vars} is $\mathcal{NP}$-hard. 
\end{mythm}
\begin{myproof}
We prove by showing that a special case of the WFL-QCOP \eqref{QCOP}-\eqref{Vars}  is
$\mathcal{NP}$-hard. Suppose that $\mathcal{E} = \emptyset$, i.e., the grid is coarse grained enough to ignore proximity constraints. Then, the constraints in Eq.~(\ref{Grid})
are dropped and we get the heaviest k-subgraph problem, which was proven to be $\mathcal{NP}$-hard~\cite{billionnet2005different}.
Since a special case of WFL-QCOP is  $\mathcal{NP}$-hard it implies that WFL-QCOP is at least $\mathcal{NP}$-hard. On the other hand,
WFL-QCOP is formulated using integer programming (Problem \ref{QCOP}), which means that it is at most $\mathcal{NP}$-hard. Therefore, we get that WFL-QCOP is $\mathcal{NP}$-hard.
\end{myproof}

%\todo{The proof is a bit awkward as it goes back-and-forth between QCOP and WFLO. Technically, the problem as defined in \eqref{QCOP}-\eqref{Vars} but, given the different (not mathematically equivalent) models of WFLO, this is different from claiming that WFLO is NPC. On the other hand, we so not want to prove that QCOP in general is NPC. So we need to get the language precise. Arik: I fixed it by referring to the problem as WFL-QCOP (and WFL-QUBO for the other one.)}

%To overcome the challenge associated with the computational hardness of WFLO,
%we reformulate the QCOP into a QUBO problem, which allows us to solve WFLO using specialized optimization hardware.  

\subsection{Solving QCOP and QUBO using Advanced Optimization Methods}

\subsubsection{QCOP solving methods}

WFL-QCOP is a strictly convex integer 
quadratically constrained optimization problem (IQCOP), which is notorious for its computational complexity~\cite{van1981another}.
%In fact, we proved in Section~\ref{sec:computational} that WFL-QCOP is $\mathcal{NP}$-hard.
The common exact methods for solving
IQCOPs are tree search \cite{land2010automatic} algorithms with a variety of sophisticated techniques to limit the search necessary to find and prove optimality.  In operations research, the generic exact approach for solving IQCOPs is the branch-and-cut based mixed integer programming (MIP) \cite{bonami2008algorithmic}. A MIP solver, such as the one we use (Gurobi), uses a linear or quadratic continuous relaxation to efficiently compute a bound on the objective function. This relaxation is used to generate cuts, implied constraints that tighten the relaxation, to identify sub-trees that can be discarded without search because they are guaranteed not to contain a solution better than the current incumbent, and to drive the heuristic branching that systematically searches for a solution. Typically, this process of cut generation, reasoning about sub-trees, and branching is performed at each node in the search tree.

Commercial MIP solvers have seen extensive development over the past 20 years and hardware-independent algorithmic improvements delivering orders of magnitude improved performance \cite{Bixby07a}. More recently the development of such solvers has focused on improved performance for quadratic problems such as WFLO-QCOP \cite{Furini19a}.
%By solving a relaxed problem to bound the objective value at each node and strengthening the relaxed problem with cutting planes to restrict the feasible region further, MIP solvers has been proved effective for attacking IQCOPs \cite{junger200950}.

%Constraint programming (CP) \cite{rossi2006handbook} is another tree search framework that can be applied to solve IQCOPs. At each node in the search tree, inference algorithms that are often encapsulated in the global constraints \cite{katriel20066}, conduct domain filtering on each constraint. Domain filtering removes possible values from variable domains by proving that a value cannot satisfy the constraint and hence cannot appear in a global solution, hence pruning sub-trees with variable assignments that are not part of any feasible solution.

%In the community of communications, IQCOPs are commonly solved via discrete ellipsoid-based search (DEBS), which is built on enumeration of integer points in the hyper-ellipsoid defining the feasible region \cite{agrell2002closest}. DEBS can be interpreted as a form of CP, where the ellipsoid geometry induces for each variable an interval domain, producing an inference algorithm removing values that do not belong to the ellipsoid. However, DEBS cannot be applied to general IQCOPs due to its special nature. This drawback has been partially overcome by Wen-Yang \cite{ku2017hybrid}, who has proposed a competitive hybrid algorithm that combines DEBS and MIP to solve IQCOPs.

% In addition to exact methods, heuristics that are designed to find good but not provably optimal solutions fast can also be used to solve IQCOPs. 



%\todo{@Chris and Jason: I still feel that we need to say how Gurobi do it (or at least how we think Gurobi does that)}

\subsubsection{Specialized Hardware for QUBO}

\todo{Eldan's part}


\section{Evaluation}
\label{sec:eval}

In this section, we 
present an empirical evaluation of the two quadratic models based on 
12 standard WFLO instances.
Our main results are: \begin{itemize}
	\item WFL-QUBO when solved using the digital annealer (DA) achieves a similar performance to state-of-the-art exact methods.
		\item The runtime for solving WFL-QUBO on the DA is two orders of magnitude lower than WFL-QCOP.  
	\item Given sufficient runtime, solving the WFL-QCOP model yields superior results on the larger instances.
\end{itemize}
We start by outlining the experimental design of our evaluation, followed by an overview of our main results,
and conclude the section with a discussion of the results and their practical implications.

\subsection{Experimental Design}



\paragraph{Standard WFLO Instances}

To test and benchmark our models, we used 12 standard benchmark WFLO instances
from the literature~\cite{MOSETTI1994105,Zhang2014,grady2005placement}. 
The 12 benchmarks result from varying 2 wind settings (1 and 36 wind directions),
2 grid sizes ($10\times10$ with cell size $200$ meters and $20\times20$ with cell size $100$ meters),
and 3 variations for number of turbines ($m \in \{20, 30 or 40\}$). 
For the $36$ wind directions setting, the velocities came from the probability distribution presented in Figure~\ref{fig:prob_wind}. We denote $\mathcal{W}$ the set of 12 instances. Note that instances with $36$ wind regimes are considered computationally harder~\cite{Zhang2014}.

Additional parameters that were used to calculate 
the velocities ($u_{id\infty}, u_{ijd}$) include: turbine height ($60$ meters),
ground roughness ($0.3$ meters), turbine rotor radius ($20$ meters), and a wind farm of size $2km \times 2km$.


\begin{figure}[t]
	\centering
	\includegraphics[scale = 0.3]{prob_wind_pdf.pdf}
	
	\caption{Wind regimes: x-axis is the angle (times 10 degrees), y-axis is the probability for wind regime, and color corresponds to free wind speed for the different wind directions.}\label{fig:prob_wind}
\end{figure}

\paragraph{Optimization Benchmarks}

We compared our two models, namely WFL-QCOP and WFL-QUBO, against two state of the art exact optimization methods:
the integer linear programming (ILP-LS) approach from~\cite{donovan2005wind}, and the 
model that uses a quadratic approximation (QP-SS)~\cite{turner2014new}. The CP approach presented in~\cite{Zhang2014} was
reported to perform worse than the rest and was left out of the experiments.
We used state of the art software, namely Gurobi~\cite{gurobi}, to solve WFL-QCOP, ILP-SS and QP-SS. 
For consistency, we shall refer to WFL-QCOP problem as QP-LS as it is a quadratic program that uses the LS objective.
In addition, we 
applied both Fujitsu's digital annealer and Gurobi to solve WFL-QUBO: we shall refer to the former as DA-LS, and to the latter
as QUBO-LS.


\paragraph{Experimental Procedure}

For exact methods (QP-SS, ILP-LS, QP-LS and QUBO-LS),
we ran Gurobi with a time limit of 3600 seconds. We collected all feasible solutions
until that point in time. The best solution in terms of energy is the solution that attained the highest objective
values (SS in QP-SS and LS in the rest of the methods).
For the digital annealer, we used a single configuration that was found to work best across the different instances. The initial temperature is 500000, the temperature mode is 0, the  interval for updating is 1, the offset increase rate is 10, the number of runs is 20, and the temperature decay rate is computed dynamically. \todo{Eldan and Arik: Do I need to write the formula for computing temperature decay rate? -- Jason} The DA ran until convergence of the underlying simulated annealing algorithm and the runtime 
of the procedure was recorded. \todo{@Chris: do we want to share the code that ran the experiments in batches? perhaps it should be separated from the problem construction and the two need to be shared separately} 

The exact methods were all solved using Gurobi's quadratic programming solver on a Ubuntu PC with Inte(R) Core(TM) i7-9700 8 core CPU @3.00 GHz with 12 MB cache and 8 GB RAM. Furthermore, QUBO-LS was solved on DA version 1 \todo{Eldan: any additional info that we need to add here?}.

\paragraph{Performance Measurement} 

We have measured the total expected energy of a solution using the SS expression (see Eq.~(\ref{eq:ss}))\footnote{Note that the total mean energy is presented in kW.}.
The only method that optimizes on SS is QP-SS. Hence, for the other methods, we converted solutions
LS-based solution into the SS expression using Eq.~(\ref{eq:ss}) as proposed in~\cite{Zhang2014}. Specifically, let $x = (x_1, \ldots, x_k), k=n^2$ be a turbine placement solution obtained by an LS-based method.
Then, the SS based energy expression can be written as, 
$$\sum_{i\in \mathcal{N}}\sum_{d \in \mathcal{D}} \frac{1}{3} x_i \Bigg(u_{id,\infty} \Bigg[1-\sqrt{\sum_{j\in \mathcal{U}_{id}}x_j\bigg( 1-\frac{u_{ijd}}{u_{id,\infty}} \bigg) } \Bigg]    \Bigg)^3.$$ 

Next, we tested the performance of our approach as function of solution time. To this end, we 
compared the average performance of all WFLO methods on all 12 WFLO instances. Since energy levels are not directly comparable between the different instances (due to varying number of turbines and wind farm specs) we used the \emph{mean relative error} (MRE) measure. The MRE is defined with respect to the best solution obtained per instance and can be computed over time. Then, the
time-varying MRE can be averaged across the 12 scenarios and presented over time. In what follows, we demonstrate the construction of the MRE measure.

Recall that $\mathcal{W}$ is the set of our 12 standard WFLO instances. 
Moreover, we let $b_w$ be the best energy solution (in terms of SS) obtained for an instance $w \in \mathcal{W}$ 
across the different methods. For example, the best solution (within 3600 seconds) for the second instance $w_2$ that aims at placing $30$ turbines on a $10\times10$ grid with a single wind regime is attained by QUBO-LS and the SS energy expression is $15742$ (kW); thus, we set $b_w = 15742$. Then, for every point in time $t \in \{0,\ldots, 3600\}$ (measured in seconds), we can compute the relative error of a given approach per instance $w$. Let $\mathcal{A}$ be the set of solution approaches, $\mathcal{A}= \{QUBO-LS, QP-LS, DA-LS, ILP-LS, QP-SS\}$. Further, denote by $b_{w,t,a}$ denote the best solution attained before time $t$ of approach $a$ for instance $w$. Then, the mean relative error at time $t$ for approach $a$ on instance $w$ is given by \begin{equation}MRE(w,t,a) = \frac{b_w - b_{w,t,a}}{b_w}.\end{equation} Note that the expression is always positive, since we are solving a maximization problem.

We can now average over the different instances at time $t$. The average performance of approach $a$ at time $t$, $MRE(t,a)$ can be computed as, \begin{equation} MRE(t,a) = \frac{1}{|\mathcal{W}|}\sum_{ w\in\mathcal{W}} MRE(w,t,a).\end{equation}
We shall present $MRE(t,a)$ for each of the approaches to demonstrate the speed of convergence of the proposed LS-based methods.


\subsection{Main Results}

In this part, we present the main findings of our empirical evaluation. 
We shall start by showing the best attained solutions by all methods.
Table~\ref{tab:results1} contains the best energy (in kW) attained by the methods
for the 12 WFLO instances after $100$ seconds of runtime for the exact methods. 
The runtime for DA-LS was always less than $10$ seconds.


\begin{table}[t!]
	
	\begin{tabular}{| c | c | c | c | c | c | c | c |}
		\toprule
		Wind Directions  & n  & m  & DA-LS  & QUBO-LS  & QP-LS  & QP-SS  & ILP-LS  \\
		\toprule
		\multirow{6}{*}{WR1}  & \multirow{3}{*}{10}       & 20       & \textbf{11185.41} & \textbf{11185.41} & \textbf{11185.41} & \textbf{11185.41} & \textbf{11185.41} \\
		& & 30   & 15740.69 & \textbf{15742.93}  & 15738.45  & 15735.10  & 15739.57     \\
		& & 40 & \textbf{19265.21} & 18952.97 & \textbf{19265.21}  & \textbf{19265.21} & \textbf{19265.21}                \\
		\cline{2-8}
		&\multirow{3}{*}{20}   & 20       & \textbf{11404.80}  & \textbf{11404.80}  & \textbf{11404.80}  & \textbf{11404.80}  & \textbf{11404.80}          \\
		&&30   & 16570.96 & 16715.19  & \textbf{16758.45}  & 16755.75 & 16695.05                  \\
		&&40   & 21874.04 & 21643.56  & 21905.50 & \textbf{21977.17} & \textbf{21977.17}        \\
		\hline
		\multirow{6}{*}{WR36} &  \multirow{3}{*}{10}    & 20       & \textbf{19221.44} & \textbf{19221.44} & \textbf{19221.44} & 19193.42 & 19022.77 \\
		&& 30  & \textbf{27450.99} & 27418.74 & 27442.90 & 27387.06 & 27247.50                     \\
		&&40   & 34888.81 & \textbf{34938.65} & \textbf{34938.65} & 34817.27  & 34671.62          \\
		\cline{2-8}
		&  \multirow{3}{*}{20}   & 20   & \textbf{19441.17}  & 19172.56 & 19336.33 & 19354.23  & 18448.64            \\
		&&30   & \textbf{27975.55} & 27443.80  & 27752.00 & 27492.11  & 26367.20                      \\
		&&40   & 35604.45 & 35127.32 & \textbf{35665.60}   & 35065.85 & 33451.94 \\
		\bottomrule                   
	\end{tabular}
	
	\vspace{0.5em}
	\caption{Total Expected Power (kW) after 100 seconds as function of benchmark and method.}\label{tab:results1}
\end{table}


\begin{table}[t!]
	
	\begin{tabular}{| c | c | c | c | c | c | c | c |}
		\toprule
		Wind Directions  & n  & m  & DA-LS  & QUBO-LS  & QP-LS  & QP-SS  & ILP-LS  \\
		\toprule
		\multirow{6}{*}{WR1}  & \multirow{3}{*}{10}       & 20       & \textbf{11185.41} & \textbf{11185.41} & \textbf{11185.41} & \textbf{11185.41} & \textbf{11185.41} \\
		& & 30   & 15740.69 & \textbf{15742.93}  & 15738.45  & 15735.1  & 15739.57     \\
		& & 40 & \textbf{19265.21} & 18952.97 & \textbf{19265.21}  & \textbf{19265.21} & \textbf{19265.21}                \\
		\cline{2-8}
		&\multirow{3}{*}{20}   & 20       & \textbf{11404.80}  & \textbf{11404.80}  & \textbf{11404.80}  & \textbf{11404.80}  & \textbf{11404.80}          \\
		&&30   & 16570.96 & 16742.66  & 16758.45  & 16755.75 & \textbf{16770.20 }                 \\
		&&40   & 21874.04 & 21832.50  & 21963.81 & \textbf{21977.17} & \textbf{21977.17}        \\
		\hline
		\multirow{6}{*}{WR36} &  \multirow{3}{*}{10}    & 20       & \textbf{19221.44} & \textbf{19221.44} & \textbf{19221.44} & 19193.42 & 19213.96 \\
		&& 30  & \textbf{27450.99} & 27442.90 & 27442.90 & 27387.06 & 27389.21                     \\
		&&40   & 34888.81 & \textbf{34938.65} & \textbf{34938.65} & 34817.27  & 34856.45          \\
		\cline{2-8}
		&  \multirow{3}{*}{20}   & 20   & \textbf{19441.17}  & 19394.58 & 19336.33 & 19354.23  & 19368.86            \\
		&&30   & \textbf{27975.55} & 27785.74  & 27752.00 & 27492.11  & 27633.20                      \\
		&&40   & 35604.45 & 35480.90 & \textbf{35665.60}   & 35065.85 & 35210.78 \\
		\bottomrule                   
	\end{tabular}
	
	\vspace{0.5em}
	\caption{Total Expected Power (kW) after 3600 seconds as function of benchmark and method.}\label{tab:results2}
\end{table}


We observe that the DA-based solution provides
comparable results to state-of-the-art exact methods. For the harder instances (of $36$ wind directions),
the three quadratic LS-based approaches (DA-LS, QUBO-LS, and QP-LS) dominate the others by finding the best (or equal) solutions in 11 out of 12 instances. For easier instances, 
the linear approach (ILP-LS) finds better solutions \todo{@Chris, Jason: do we have a good explanation for this?}. Table~\ref{tab:results2} shows the same type of results as in Table~\ref{tab:results1}, but now after $3600$ seconds. We see that the DA is still competitive with the other approaches (recall, its solution was found after $10$ seconds),
while QP-LS finds the best solutions in 10 out of the 12 scenarios. For the 6 hard scenarios (12 wind directions), our quadratic solutions
dominate all 6. \todo{@Chris: I am not sure that the first table contributes anything beyond the second table}

We now turn to present the mean relative error over time per approach, i.e., we are presenting $MRE(a,t)$). Note that lower values indicate better performance.  Figure~\ref{fig:results1} 
presents the MRE after $50$ seconds of running the solutions focusing runtime performance, while Figure~\ref{fig:results2} corresponds
to the full run (after $3600$ seconds). The latter provides insights into best performance in terms of energy. The 5 different lines (and their colors) 
correspond to the 5 WFLO solutions. The horizontal axis represents time in seconds.


The results show that on average (across the 12 instances),
the ILP-LS method starts with poor solutions, yet ends up improving over QP-SS and QUBO-LS (after around 1250 seconds).

The two methods that are based on advanced software (QP-LS) and hardware (DA-LS) solutions
outperform the rest both in terms of runtime (best results in short time), and in terms of overall performance.
Specifically, for most instances, QP-LS gives the best solution at 3600 seconds, while the DA-based approach
yields competitive solutions after less than 10 seconds. Furthermore, QP-LS converges to the best solution faster than all other exact methods. In fact, it is inferior only to the approximate DA-LS method, which is geared mainly toward runtime. This is an encouraging result, especially due to the pessimistic computational complexity of WFLO.

\begin{figure}[t]%
	
	\includegraphics[width=\textwidth]{energy_50.png}  
	\caption{Mean Relative Energy (wrt best feasible solution) vs. Runtime (after 50 seconds).}%
	
	\label{fig:results1}%
\end{figure}

\begin{figure}[t]%
	
	\includegraphics[width=\textwidth]{energy_3600.png}   \\%\\
	\caption{Mean Relative Energy (wrt best feasible solution) vs. Runtime (after 3600 seconds).}%
	
	\label{fig:results2}%
\end{figure}



\subsection{Discussion and Practical Implications}

Our result show empirically that the quadratic programming models that we
introduced in Section~\ref{sec:QUBO4WFLO} (WFL-QCOP and WFL-QUBO) are useful in two ways.
First, the two problems are the basis for the QP-LS and QUBO-LS methods that couple the two
quadratic formulations of WFLO with Gurobi optimizer. This turns out
to be a powerful method as it gives the best performance after 3600 seconds for most instances (7 out of 12),
and especially,
for the hard instances with 36 wind directions. In addition, the runtime performance of QP-LS is second
only to the digital annealer.  Second, the QUBO model can be mounted onto a quantum-inspired optimization hardware (e.g., Fujitsu's digital annealer) demonstrated by the DA-LS approach in our evaluation. DA-LS yields the best results in 7 out of 12 
instances, shows competitive performance on the other instances, and outperforms all methods in terms of runtime.
 
The ability of DA-LS to quickly obtain competitive solutions on large WFLO instances 
is crucial when solving
multiple wind farm layout problems. Moreover, it allows the flexibility to tune the various WFLO parameters (including turbine specs, number of turbines, and farm sizes) without incurring a steep computational cost that came
with (re-)solving the problem using exact methods. In addition, the performance loss due to the use of an approximate method 
that exploits advanced technology is negligible. This balance (instead of trade-off) between
runtime and total energy, points at the potential contribution of the proposed quadratic framework for solving WFLO. 


  
\section{Conclusion}
\label{sec:conclusion}

In this work, we presented a novel quadratic programming framework
for solving 
wind farm layout optimization. 
The framework is based on two models: a constrained model (WFL-QCOP) and an unconstrained binary
model (WFL-QUBO). The former can be used to solve WFLO using advanced optimization software,
while the latter can be mounted onto quantum-inspired machines that can quickly (approximately) solve WFLO. We used the constrained quadratic formulation to prove 
that the computational complexity of WFLO is $\mathcal{NP}$-hard, a pessimistic result that 
was also noticed experimentally in~\cite{Zhang2014}. However,
our empirical evaluation shows that when exact solvers like Gurobi~\cite{gurobi} are
coupled with our WFL-QCOP and WFL-QUBO models they are able to solve small WFLO instances to optimality in less than 3600 seconds.
We also show that when an advanced hardware is used with the QUBO representation, large instances WFLO can be solved quickly with no significant degradation in the total energy compared to exact methods.
This runtime improvement and state-of-the-art performance in terms of the total energy of the resulting solutions makes our quadratic framework useful for designing new wind farms, while varying multiple input parameters. 


In future work, we aim at extending our quadratic models to include additional 
WFLO aspects, such as noise, non-flat terrain and multiple objectives. 
Moreover, we wish to explore real-world wind farm design problems 
that may be larger and more complex than the 12 standard instances 
and solve them 
using our quadratic programming approach. 

\bibliographystyle{elsarticle-num-names}
%\bibliographystyle{plainnat}
\bibliography{Biblio}

\end{document}



%
%\begin{table}[t]
%	
%
%	\label{tab:results1}
%	\vspace{-0.5em}
%	\scriptsize
%	\centering
%	\begin{tabular}{ c @{\hspace{1em}} c @{\hspace{1em}} c @{\hspace{1em}} c}
%		\toprule
%		Paper & Exact Optimization & Discrete Grid & Flat Terrain  \\
%		\midrule
%		\citet{Zhang2014,turner2014new} & + & + & +  \\
%		\citet{kuo2016wind} & + & + & -  \\
%		 \citet{} & + & - & + \\
%		
%		\bottomrule
%	\end{tabular}
%	\vspace{-1em}
%	\caption{Literature review of Wind Farm Optimization Layout solutions.}
%		
%\end{table}

© 2020 GitHub, Inc.
Terms
Privacy
Security
Status
Help
Contact GitHub
Pricing
API
Training
Blog
About
