\documentclass[preprint,12pt]{elsarticle}
\usepackage{etex}

\makeatletter
\providecommand{\doi}[1]{%
	\begingroup
	\let\bibinfo\@secondoftwo
	\urlstyle{rm}%
	\href{http://dx.doi.org/#1}{%
		doi:\discretionary{}{}{}%
		\nolinkurl{#1}%
	}%
	\endgroup
}
\makeatother
%\usepackage{natbib}
% encoding and languages
\usepackage{hyperref}
\usepackage[utf8]{inputenc}
\usepackage[T1]{fontenc}
\usepackage[ngerman,english]{babel}
\usepackage{wrapfig}
\usepackage{times}
\usepackage{paralist}
\usepackage{graphicx, color}
%\usepackage{subfigure}
\usepackage{subfig}
\usepackage{subfloat}
\usepackage{booktabs}
\usepackage{caption}
%\usepackage{subcaption}
\usepackage{paralist}
\usepackage{amssymb, amsfonts}
\usepackage{amsmath}
%\usepackage{amsthm}
\usepackage{amsopn}
\newtheorem{myProb}{Problem}
\newtheorem{mythm}{Theorem}
\newtheorem{myproof}{Proof}
\newtheorem{mysketch}{Proof Sketch}
%\usepackage[ruled]{algorithm2e}
%\usepackage[colorinlistoftodos]{todonotes}
%\usepackage{qtree}
\usepackage{tikz, tikz-qtree}
\usepackage[ruled]{algorithm2e}
%\usepackage{Algorithmic}
\usepackage{times}
\usepackage{microtype}
\usepackage{url}
\usepackage{balance}
\let\proof\relax 
\let\endproof\relax
\usepackage{multirow}
\usepackage{relsize}
\usepackage{caption}
\usepackage{color, colortbl}
\usepackage{array, booktabs, ragged2e}
\usepackage{makecell}
\usepackage{tikz}
\usepackage{textpos}

\usepackage{pdflscape}
\newcommand{\RNum}[1]{\uppercase\expandafter{\romannumeral #1\relax}}

\graphicspath{{./Figures/}}

\newcommand{\mypar}[1]{\smallskip\noindent\textbf{#1.}}

\newtheorem{prob}{Problem}


\newtheorem{mydef}{Definition}
%\newtheorem{remark}{Remark}

\newcommand{\todo}[1]{{\textcolor{red}{\bf {#1}}}}

%\newcommand{\mytitle}{Exploiting Software and Hardware Advances  in Optimization Methods for Improved Wind Farm Design}
\newcommand{\mytitle}{Exploiting Hardware and Software Advances 
 for Quadratic Models of Wind Farm Layout Optimization}
%\newcommand{\mytitlerunning}{\mytitle}
%\newcommand{\myauthor}{Arik Senderovich, Eldan Cohen, and J. Christopher Beck}
%\newcommand{\myauthorhyperref}{Arik Senderovich, Eldan Cohen, and J. Christopher Beck}


%\hypersetup{
%    bookmarks=true,        % show bookmarks bar?
%    unicode=true,          % non-latin characters in bookmarks?
%    pdffitwindow=false,    % window fit to page when opened
%    pdfstartview={FitH},   % fits width of the page to the window
%    pdftoolbar=true,       % show acrobat toolbar?
%    pdfmenubar=true,       % show acrobat menu?
%    colorlinks=true,      % false: boxed links; true: color links
%    linkcolor={blue!80!black},       % color of internal links
%    urlcolor={blue!30!black},        % color of external links
%    citecolor={green!60!black},       % color of links to bibliography
%    filecolor=black,       % color of file links
%    pdftitle={\mytitle},           % title
%    pdfauthor={\myauthorhyperref}, % author
%    pdfcreator={\myauthorhyperref} % creator of pdf
%}

\begin{document}
	

\title{\mytitle}
%\subtitle{Extended Abstract}
%\author{}
%\institute{}
%\author{\myauthor}
%\institute{Department of Mechanical and Industrial Engineering, University of Toronto\\
%\email{sariks@mie.utoronto.ca, ecohen@mie.utoronto.ca, jcb@mie.utoronto.ca}}


\author[add1]{Arik Senderovich}

%\ead[url]{https://ischool.utoronto.ca/profile/arik-senderovich/}
\ead{arik.senderovich@utoronto.ca}

\author[add2]{Eldan Cohen}
\ead{ecohen@mie.utoronto.ca}
%\ead[url]{homepageldan}




%\address[1]{}



%\address[2]{40 St. George St., Toronto, Canada}
\author[add2]{Jiachen Zhang}
\ead{jasonzjc@mie.utoronto.ca}

\author[add2]{J. Christopher Beck\corref{cor1}}
\ead{jcb@mie.utoronto.ca}


%\ead[url]{homepageldan}
%\address[3]{}
\address[add1]{Faculty of Information, University of Toronto, 140 St. George St., Toronto, Canada}
\cortext[cor1]{Corresponding author}
\address[add2]{Department of Mechanical and Industrial Engineering, University of Toronto, 40 St. George St., Toronto, Canada}

\begin{abstract}
%Wind farms play an increasingly important
%role as a source of renewable energy and 
The wind farm layout optimization (WFLO) problem concerns
selecting turbine positions to maximize energy production. While a number of heuristic and exact techniques have been proposed in the literature to solve WFLO, it is natural to model the interaction between turbines as a quadratic constraint as, indeed, has been done in some existing optimization models. Recent advances in optimization hardware and software have both targeted quadratic constraints: commercial mixed integer linear solvers have been extended to quadratic problems and nascent specialized hardware, including quantum computing systems, have focused on solving Quadratic Unconstrained Binary Optimization (QUBO) problems. 
%It is a key problem 
%when designing a new wind farm, as it has a direct influence
%on the amount of the total produced energy. 
%In recent years there have been advances in both software technologies (e.g., the extension of commercial mixed integer linear programming solvers to handle quadratic problems) and nascent hardware development (e.g., quantum computing chips and specialized classical chips) that have relevance for the WFLO problem. In both cases, the advances have 
%Recent years saw 
%advances in software and hardware optimization technologies, such as Gurobi's quadratic solver 
%and Fujitsu's digital annealer (DA), respectively. 
%These technologies create an opportunity to improve 
%over current wind farm optimization solutions
%both in terms of solution quality and solve time. 
%In order to exploit these technologies, 
%we must represent the WFLO problem as 
%a quadratic optimization model. Specifically, in order 
%to apply a software-based optimization
%solution (Gurobi), 
In this paper, we 
introduce two novel quadratic programming models for WFLO: a quadratic constrained optimization problem (QCOP) with binary decision variables and a QUBO.
%Subsequently, we provide the unconstrained counterpart of the QCOP, 
%namely the quadratic unconstrained binary optimization (QUBO), which  
%is the underlying representation for quantum-inspired 
%optimization hardware such as the DA. 
A thorough empirical 
evaluation using a commercial solver and a specialized classical chip for QUBO solving demonstrates the strengths of the two models. 
Our results show that both approaches yield fast and high-quality solutions 
and
improve over existing WFLO solutions. In particular, the QUBO model excels at finding high quality solutions very quickly while the QCOP model is able to find better solutions and provide quality guarantees over a longer run-time.
\end{abstract}
\begin{keyword}
	Wind Farm Layout Optimization (WFLO) \sep Quadratic Programming (QP) \sep Quadratic Unconstrained Binary Optimization (QUBO) \sep Digital Annealer
\end{keyword}

\maketitle 

\section{Introduction}

We revisit 
the wind farm layout optimization (WFLO) problem 
that was first considered 
in~\citet{MOSETTI1994105}. 
In WFLO, we aim to place wind turbines 
in a predefined area to maximize 
the total generated power. 
Turbine location decisions    
influence
the
total power production due to  
interference effects, aka wakes, generated by upstream 
turbines. Furthermore, 
there are proximity constraints between turbines
that prevent from turbines to be placed next to each other.
These inter-turbine effects 
are naturally captured using quadratic constraints and objective
functions. Several exact and heuristic optimization approaches have 
been proposed to solve
the WFLO problem~\cite{turner2014new,Zhang2014}. However, these
approaches represented the quadratic effects and constraints
using linear models that approximate the real dependencies. 

In this work, we introduce two 
novel quadratic WFLO models. The first model 
represents WFLO using a 
quadratic constrained optimization problem (QCOP). 
The QCOP can be solved using state-of-the-art solvers that
specialize in such problems, e.g., Gurobi~\cite{guorbi}. 
Our second model is based on a quadratic unconstrained binary optimization (QUBO) representation of the WFLO 
problem.
This enables the use of specialized optimization hardware tailored to solve
QUBOs. In this paper, we use Fujitsu's digital annealer to demonstrate the strength
of the approach. The main contributions of this paper are: 
\begin{enumerate} 
\item We propose two novel quadratic WFLO models that capture wake and proximity
effects directly without going through a linear approximation, and, 
\item We test the two formulations
on a commerical software-based quadratic solver (Gurobi) and on Fujitsu's digital 
annealer. 
\end{enumerate} Our experiments are focused on the performance of the two approaches
compared to 
existing optimization methods. To this end,
we solve 12 standard WFLO benchmark problems (c.f.~\cite{turner2014new}). 
We show, empirically, that solving quadratic WFLO models
using Gurobi and the DA
achieves state-of-the-art performance, often outperforming
existing solutions when provided sufficient run-time. Furthermore, we
we observe that the digital 
annealer often provides order of magnitude faster feasible solutions
compared to our QCOP based approach.

The remainder is structured as follows \todo{adjust structure}. 


\section{Background}
\label{sec:related}

In this part, we first present the
physical model of a wind farm that we use throughout the paper. 
Subsequently, we first provide an overview of 
existing mathematical
optimization methods for WFLO, focusing
on their similarities and differences
compared to our approach.
 
\subsection{Wind Farm Model}

We consider a square shaped wind farm that we model 
as an $N\times N$ grid
with $N$ being the number of cells in each axis. 
Each cell has the area of $c^2$ (in meters), i.e.,
if $c=100$ meters and $N=20$, the farm area will be $2km\times 2km$.
In this work, we assume that the terrain is flat and that 
we have no constraints related to the terrain.
Furthermore, we assume a sterile environment and do not consider the influence of noise
on the wind farm's surroundings. The wind farm literature treats the more complex
cases of non-flat terrain and environmental considerations in~\cite{} and~\cite{}, respectively.
In our work, we introduce a novel optimization approach that can be extended,
in the future, to include 
such considerations. 
 
In principle, a higher number of turbines placed in the wind farm produces more energy. However,
we must consider two types of interference between the turbines. 

First, 
turbines cannot be placed at high proximity to each other. 
Specifically, the turbines must be placed at least five rotor diameters apart from each other. Depending on the cell size, we shall formulate the corresponding constraints as part of the optimization model. 
For example, assuming that a turbine is placed
at the center of the cell (cell size of $100$ meters) and for rotor diameter of $40$ meters, 
we must ensure that two turbines will not placed in adjacent cells. 

Second, turbines influence each other through an effect called \emph{wake}. Wakes cause reduction of the effective power produced by a turbine due to upstream turbines changing airflow dynamics.

\todo{Explain the wake model - single wake and linear superposition. Explain that it is less accurate but leads to a quadratic objective. We then assess it using sum of squares. Explain that there are $m$ turbines to be placed. use Zheng and use Mosetti.}

\todo{Wind direction}

 
\subsection{Optimization Methods for WFLO}

The allocation of turbines in a wind farm was first considered 
as an optimization problem in~\cite{MOSETTI1994105}. Since
then, WFLO has caught the attention of optimization researchers. 
The WFLO problem can either be solved \emph{exactly} using
techniques from operations research (e.g., linear programming, 
mixed-integer programming, stochastic programming, and constraint programming)\cite{Zhang2014,turner2014new}
or solved \emph{approximately} using various meta-heuristics (e.g., genetic algorithms and local search) \cite{MOSETTI1994105}.
The main advantage of the former techniques
is that they provide guarantees on solution quality, 
while the latter methods generate 
quick and sufficiently accurate solutions. In this work, we compare 
exact and approximate methods for WFLO. Both use 
quadratic optimization models... \todo{add more variations and explain that we
	focus on one to demonstrate the usefulness of quadratic models}


\section{Quadratic Models for WFLO}
\label{sec:QUBO4WFLO}

In this part, 
we formulate the WFLO problem
as two quadratic optimization problems: a
quadratic constrained optimization (QCOP) problem and 
a quadratic unconstrained binary optimization (QUBO) problem. 
To this end,
we follow standard WFLO definitions
and notation presented in~\cite{Zhang2014}. First, we present WFLO's inputs and decision variables.
Subsequently, we present the mathematical formulation of the two problems. 

\paragraph{WFLO Inputs and Decision Variables} Let $x_i$ be a binary decision variable that represents whether a 
turbine is positioned at location $i \in \mathcal{N}$ with $\mathcal{N}$ being the set 
of $n$ possible locations and 
$m$ being the total number of required turbines. Note that we assume a discrete grid with 
a finite number of possible positions. We 
write $x$ as shorthand for $x = (x_1,\ldots, x_n)$. Furthermore,
we denote by $\mathcal{E} \subseteq \mathcal{N}\times \mathcal{N}$
the set of location pairs that cannot simultaneously host turbines 
due to proximity constraints. 
Let $u_{id}$ and $u_{id, \infty} $ 
be the wind speeds at turbine location $i$ for wind regime $d \in \mathcal{D}$
with and without interference from other turbines due to wake effects, respectively. A wind regime is a combination 
of wind speed and direction. 
The probability of wind regime 
$d$ is given by $p_d$ with $\sum_{d \in \mathcal{D}}^{} p_d = 1$. For example, 
regime $d=(10deg, 12km/h)$ corresponds to a north-eastern wind.

 
Next, we denote by $\mathcal{U}_{id} \subseteq \mathcal{N}$ 
the set of downstream 
turbine locations 
given that a turbine is placed at cell $i$ for wind regime $d$ (i.e., turbines placed in these locations 
will be influenced by a turbine placed at $i$). Finally,
we denote by $u_{ijd}$ the wind speed at turbine $j$ due to a single wake from upstream turbine $i$ with $j \in \mathcal{U}_{id}$. 
The following objective function  
represents the total expected energy of a wind farm given placement decisions 
$x$: \begin{equation}
f(x) = \sum_{i \in \mathcal{N}}^{} \sum_{d \in \mathcal{D}}^{} p_d ( \frac{1}{3} \ u_{id, \infty}^3 x_i  - \sum_{j \in \mathcal{U}_{id}}^{} \frac{1}{3}(u_{id, \infty} ^3 - u_{ijd}^3)x_i x_j)   \label{ObjFunc}\\
%&s.t.:& \nonumber\\
%&\mbox{       }& \sum_{i \in \mathcal{N}}^{} x_i = m,\label{Cardinality}\\
%&\mbox{       }& x_i + x_j \leq 1,   \forall (i,j) \in \mathcal{E}, \label{Grid}\\
%&\mbox{       }& x_i \in \{0,1\},     \forall i \in \mathcal{N}.
\end{equation} 

\subsection{WFLO as QCOP}

To 
maximize $f(x)$ and satisfy the number of turbines and proximity 
constraints we can 
the following quadratic optimization problem (QCOP):
\begin{eqnarray} \label{QCOP}
&\max_{x_i}^{}& \sum_{i \in \mathcal{N}}^{} \sum_{d \in \mathcal{D}}^{} p_d ( \frac{1}{3} \ u_{id, \infty}^3 x_i  - \sum_{j \in \mathcal{U}_{id}}^{} \frac{1}{3}(u_{id, \infty} ^3 - u_{ijd}^3)x_i x_j)   \\
&s.t.:& \nonumber\\
&\mbox{       }& \sum_{i \in \mathcal{N}}^{} x_i = m,\label{Cardinality}\\
&\mbox{       }& x_i + x_j \leq 1,   \forall (i,j) \in \mathcal{E}, \label{Grid}\\
&\mbox{       }& x_i \in \{0,1\},     \forall i \in \mathcal{N}.
\end{eqnarray} 

Solving QCOP with Gurobi. Explain.



\subsection{WFLO as QUBO}

The equivalent QUBO formulation is given by,
\begin{equation}\max_{x}^{} f(x) - \lambda_1 (\sum_{i \in \mathcal{N}}^{} x_i -m) ^2 - \lambda_2 \sum_{(i,j) \in \mathcal{E}}^{} x_i x_j , \label{QUBO}\end{equation}
with the term multiplied by $\lambda_1$ 
ensures that exactly $m$ turbines are placed, while the term multiplied by $\lambda_2$ penalizes
solutions that place turbines in adjacent locations. The two penalty terms $\lambda_1$ and $\lambda_2$ set large enough to ensure constraint satisfaction (see Section~\ref{sec:theory}). The QUBO in Eq. (\ref{QUBO}) allows us
to solve the WFLO problem using quantum-inspired hardware.  



\section{Evaluation}
\label{sec:eval}


\subsection{Experimental Design}



\paragraph{WFLO Instances}

\paragraph{Optimization Benchmarks}

\paragraph{Experimental Setting and Implementation Details}

\subsection{Main Results}



\subsection{Discussion \& Limitations}



\section{Conclusion}
\label{sec:conclusion}

\bibliographystyle{elsarticle-num-names}
%\bibliographystyle{plainnat}
\bibliography{Biblio}

\end{document}



%
%\begin{table}[t]
%	
%
%	\label{tab:results1}
%	\vspace{-0.5em}
%	\scriptsize
%	\centering
%	\begin{tabular}{ c @{\hspace{1em}} c @{\hspace{1em}} c @{\hspace{1em}} c}
%		\toprule
%		Paper & Exact Optimization & Discrete Grid & Flat Terrain  \\
%		\midrule
%		\citet{Zhang2014,turner2014new} & + & + & +  \\
%		\citet{kuo2016wind} & + & + & -  \\
%		 \citet{} & + & - & + \\
%		
%		\bottomrule
%	\end{tabular}
%	\vspace{-1em}
%	\caption{Literature review of Wind Farm Optimization Layout solutions.}
%		
%\end{table}

